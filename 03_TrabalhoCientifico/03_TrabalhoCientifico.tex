\documentclass[t]{beamer}

% Load general definitions
\input{preamble.tex}

% Specific definitions
\title[]{Metodologia Científica}
\subtitle[]{Estruturação do trabalho científico}
\author[]{Patrícia Lucas\\{\footnotesize }}
\institute{Bacharelado em Sistemas de Informação \\ IFNMG  - Campus Salinas}
\date{\scriptsize Salinas\\Fevereiro 2021}

\begin{document}

% cover page
\setbeamertemplate{footline}{}
\begin{frame}

\begin{center}
\includegraphics[width=.15\textwidth]{}
\end{center}
  \titlepage
  \begin{tikzpicture}[remember picture,overlay]
  \node[anchor=south east,xshift=-5pt,yshift=5pt] at (current page.south east) {\tiny Versão 1.2021};
  \node[anchor=south west,yshift=0pt] at (current page.south west) {\includegraphics[width=.25\textwidth]{Logos/salinas_horizontal_jpg.jpg}};
  \end{tikzpicture}  
\end{frame}

% Main slides
\begin{ftst}{A escolha do objetivo da pesquisa}{Estruturação do trabalho científico}
\justifying

O segredo de um trabalho de pesquisa de sucesso consiste em ter um bom objetivo. Uma vez definido o objetivo do trabalho, tudo o mais gravita em redor dele. 
\vone

\begin{figure}
    \centering
    \includegraphics[scale=0.13]{Figuras/03_objetivos.png}
    \label{fig:objetivos}
\end{figure}

\end{ftst}

%=====

\begin{ftst}{A escolha do objetivo da pesquisa}{Estruturação do trabalho científico}
\justifying
\textbf{O caminho para a escolha de um objetivo de pesquisa:}
\vone
\begin{itemize}
    \item Escolher um tema de pesquisa, ou seja, uma área de conhecimento na qual vai trabalhar.
    \item Realizar a revisão bibliográfica. A não ser que o autor já seja especialista na área escolhida, ele vai precisar ler muitos trabalhos já publicados nessa área para saber o que está sendo feito (estado da arte) e o que ainda precisa ser feito (problemas em aberto).
    \item Definir o objetivo de pesquisa. Uma vez feita a revisão bibliográfica, o objetivo de pesquisa possivelmente será fortemente relacionado com um dos problemas em aberto verificados no passo anterior.
\end{itemize}

\end{ftst}

%=====


\begin{ftst}{A escolha do objetivo da pesquisa}{Estruturação do trabalho científico}
\justifying
\textbf{O tema:}
\vone
\begin{figure}
    \centering
    \includegraphics[scale=0.6]{Figuras/03_temas.jpg}
    \label{fig:temas}
\end{figure}
\vone
 Quanto mais amplo o tema, maior a quantidade de livros e artigos que terão de ser lidos. Portanto, recomenda-se buscar temas cada vez mais específicos antes de propor um objetivo de pesquisa.

\end{ftst}

%=====

\begin{ftst}{A escolha do objetivo da pesquisa}{Estruturação do trabalho científico}
\justifying
\textbf{O problema de pesquisa:}
\vone
Uma monografia deve apresentar uma solução para um problema. Inicialmente, portanto, um problema deve ser identificado.


\end{ftst}

%=====

\begin{ftst}{A revisão bibliográfica}{Estruturação do trabalho científico}
\justifying
A revisão bibliográfica deve ser muito bem planejada e conduzida.
\vone
Pode-se iniciar a pesquisa com uma leitura de trabalhos mais abrangentes que deem uma visão do todo para depois ir se aprofundando cada vez mais em temas cada vez mais específicos.
\vone
Quando se faz uma pesquisa em que alguma técnica de computação é aplicada a alguma outra área do conhecimento, é necessário fazer a revisão bibliográfica sobre a técnica em si sobre a área de aplicação e, mais do que tudo, sobre as aplicações que já foram tentadas com essa técnica ou com técnicas semelhantes na mesma área ou em áreas equivalentes.


\end{ftst}

%=====

\begin{ftst}{A revisão bibliográfica}{Estruturação do trabalho científico}
\justifying
\textbf{Fichas de leitura:}
\vone
Durante todo o processo de leitura, é fundamental que sejam feitas anotações.
\vone
Conceitos-chave e ideias novas devem ser anotados sempre que forem detectados na leitura. 
\vone
Em geral, inicia-se uma ficha de leitura, seja em papel, seja no computador,
escrevendo a referência bibliográfica da obra consultada. Em seguida são feitas as anotações relevantes.


\end{ftst}

%=====

\begin{ftst}{A revisão bibliográfica}{Estruturação do trabalho científico}
\justifying
\textbf{Tipos de fontes bibliográficas:}
\vone
\begin{enumerate}
    \item Livros: apresentar determinada área da ciência de forma didática e bem fundamentada.
    \item Artigos de eventos ou periódicos: ideias de pesquisa.
\end{enumerate}


\end{ftst}

%=====

\begin{ftst}{A revisão bibliográfica}{Estruturação do trabalho científico}
\justifying
\textbf{O que deve ser necessariamente lido:}
\vone
\begin{enumerate}
    \item Livros e artigos de revisão: esse tipo de bibliografia apresenta ao pesquisador o estado
da arte da área de pesquisa e sua evolução histórica.
    \item Trabalhos clássicos: normalmente são destacados nos artigos de revisão.
    \item Fontes mais recentes sobre o assunto da pesquisa: é fundamental que um trabalho de pesquisa tenha como referência também os desenvolvimentos mais recentes na área.
\end{enumerate}


\end{ftst}

%=====

\begin{ftst}{A revisão bibliográfica}{Estruturação do trabalho científico}
\justifying
\textbf{Como sistematizar a pesquisa bibliográfica:}
\vone
\begin{enumerate}
    \item[a.] Listar os títulos de periódicos e eventos relevantes para o tema de pesquisa e os títulos de periódicos gerais em computação que eventualmente possam ter algum artigo na área do tema de pesquisa.
    \item[b.] Obter a lista e todos os artigos publicados nos últimos cinco anos (ou mais) nesses veículos.
    \item[c.] Selecionar dessa lista aqueles títulos que tenham relação com o tema de pesquisa.
    \item[d.] Ler o \textit{abstract} desses artigos e, em função da leitura, classificá-los como relevância “alta”, “média” ou “baixa”.
\end{enumerate}


\end{ftst}

%=====

\begin{ftst}{A revisão bibliográfica}{Estruturação do trabalho científico}
\justifying
\textbf{Como sistematizar a pesquisa bibliográfica:}
\vone
\begin{enumerate}
    \item[e.] Ler os artigos de alta relevância e fazer fichas de leitura anotando os principais conceitos e ideias aprendidos. Anotar também títulos de outros artigos possivelmente mencionados na bibliografia de cada artigo (mesmo que com mais de cinco anos) e que pareçam relevantes para o trabalho de pesquisa. Incluir esses artigos na lista dos que devem ser lidos (inicialmente o \textit{abstract} e, se for relevante, o artigo todo).
    \item[f.] Dependendo do caso, ler também os artigos de relevância média e baixa, mas iniciando sempre pelos de alta relevância.
\end{enumerate}


\end{ftst}

%=====

\begin{ftst}{A revisão bibliográfica}{Estruturação do trabalho científico}
\justifying
\textbf{Como sistematizar a pesquisa bibliográfica:}
\vone
Depois do último passo, o aluno poderá decidir:
\vone
\begin{enumerate}
    \item[1.] Se já tem material suficiente para elaborar uma ideia de pesquisa consistente.
    \item[2.] Se precisa expandir a pesquisa examinando artigos mais antigos (expandindo o passo b) ou periódicos menos relevantes (expandindo o passo a).

\end{enumerate}


\end{ftst}

%=====

\begin{ftst}{O Objetivo}{Estruturação do trabalho científico}
\justifying
O objetivo da pesquisa deve ser diretamente verificável ao final do trabalho. Um bom objetivo de pesquisa possivelmente demonstrará que alguma hipótese sendo testada é ou não verdadeira.
\vone
Portanto, o objetivo geral e os objetivos específicos do trabalho devem ser expressos na forma de uma condição não trivial cujo sucesso possa vir a ser verificado ao final do trabalho. Um objetivo bem expresso, em geral, terá verbos como “demonstrar”, “provar”, “melhorar” (de acordo com alguma métrica definida) etc.


\end{ftst}

%=====

\begin{ftst}{O Objetivo}{Estruturação do trabalho científico}
\justifying
\textbf{A extensão do objetivo de pesquisa:}
\vone
Alunos de graduação e pós-graduação devem atingir os objetivos colocados dentro do tempo regulamentar que os seus cursos estabelecem e, portanto, a complexidade desses objetivos deve ser consistente com esse tempo.
\end{ftst}

%=====

\begin{ftst}{O Objetivo}{Estruturação do trabalho científico}
\justifying
\textbf{Objetivo de pesquisa versus objetivo técnico:}
\vone
É aceitável que um trabalho de graduação e mesmo de especialização tenha objetivos técnicos, ou seja, espera-se nesses graus que os alunos sejam capazes de demonstrar que aprenderam determinados conceitos e conseguem colocá-los em prática. 
\vone
Assim, é aceitável que um aluno de graduação, ao final de seu curso, desenvolva um sistema usando conceitos aprendidos durante o curso e que apresente o sistema como trabalho final. 

\end{ftst}

%=====

\begin{ftst}{O Objetivo}{Estruturação do trabalho científico}
\justifying
\textbf{Os objetivos específicos:}
\vone
Os objetivos específicos devem ser escolhidos da mesma forma que o objetivo geral, ou seja, devem ser não triviais e verificáveis ao final do trabalho. Normalmente, os objetivos específicos não são etapas do trabalho, mas subprodutos. 
\vone
Deve-se tomar cuidado para não confundir os objetivos específicos com os passos do método de pesquisa. 

\end{ftst}

%=====

\begin{ftst}{O método de pesquisa}{Estruturação do trabalho científico}
\justifying
Em geral, as monografias têm um capítulo ou seção designado “metodologia”. Entretanto, metodologia seria o estudo dos métodos. Apesar do uso corrente, linguisticamente seria mais correto afirmar que um trabalho científico individualmente tem um método de pesquisa e não uma metodologia. 
\vone
O método de um trabalho científico só pode ser estabelecido depois que o objetivo tiver sido definido. Por esse motivo, a revisão bibliográfica não deveria nem fazer parte do método.
\vone
A etapa de revisão bibliográfica é um pré-requisito para a realização do trabalho de pesquisa, pois quem não estudou o assunto não tem como propor um objetivo válido.

\end{ftst}

%=====

\begin{ftst}{O método de pesquisa}{Estruturação do trabalho científico}
\justifying
O método consiste na sequência de passos necessários para demonstrar que o objetivo proposto foi atingido, ou seja, se os passos definidos no método forem executados, os resultados obtidos deverão ser convincentes.
\vone
O método deve então indicar se protótipos serão desenvolvidos, se modelos teóricos serão construídos, quais experimentos eventualmente serão realizados, como os dados serão organizados
e comparados, e assim por diante, dependendo do tipo de trabalho.


\end{ftst}

%=====

\begin{ftst}{O método de pesquisa}{Estruturação do trabalho científico}
\justifying
Descrever um conjunto de passos que constitua um método de trabalho científico aceitável exige alguns conhecimentos sobre o método científico.
\vone
Coisas como “trabalhar com dois grupos, um com a ferramenta e outro sem a ferramenta” até poderia
ser parte de um método, mas não é suficiente. 
\vone
Se a diferença entre as médias dos dois grupos for de $0.5$ ponto percentual, pode-se concluir que um grupo foi melhor que o outro? Ou pode ter sido obra do acaso? E se a diferença for de cinco pontos percentuais?
\vone
Como saber? Existem algumas informações trazidas pela estatística que devem ser do conhecimento de qualquer pessoa que se aventure a desenvolver pesquisa científica.

\end{ftst}

%=====

\begin{ftst}{O método de pesquisa}{Estruturação do trabalho científico}
\justifying
\textbf{Dados versus conceitos}
\vone
O método de pesquisa não consiste apenas em coletar dados para suportar a hipótese de trabalho. 
\vone
Trabalhos acadêmicos que se restringem à realização de pesquisas de opinião através de questionários com a consequente tabulação dos dados e apresentação de gráficos não terão validade se não trouxerem consigo alguma informação nova.
\vone
Exemplo: há algum tempo pesquisadores realizaram uma pesquisa na Inglaterra, onde entrevistaram homens e mulheres perguntando quantos parceiros sexuais haviam tido ao longo da vida.
\end{ftst}

%=====

\begin{ftst}{O método de pesquisa}{Estruturação do trabalho científico}
\justifying
\textbf{Definições constitutivas e operacionais}
\vone
As definições constitutivas são definições de dicionário. 
\vone
As definições operacionais atribuem significado a um constructo ou variável especificando as atividades ou ‘operações’ necessárias para medi-lo ou manipulá-lo.”
\vone
Exemplo: o termo “facilidade” pode ser definido como o número de toques no teclado ou mouse para realizar determinada tarefa.
\vone
O importante aqui é enfatizar que, no caso de variáveis que representem características não formais, é necessário utilizar definições operacionais para que o fenômeno associado à variável possa efetivamente ser medido. Sem isso o trabalho seria apenas especulativo.
\end{ftst}

%=====

\begin{ftst}{O método de pesquisa}{Estruturação do trabalho científico}
\justifying
\textbf{Variáveis}
\vone
Uma variável é um nome que se dá a um fenômeno que pode ser medido e que varia conforme a medição. Se não variasse, seria uma constante e não teria maior
interesse para a pesquisa.
\vone
\textbf{Variáveis \textbf{contínuas} versus variáveis \textit{discretas}}.
\vone
Variáveis \textbf{categóricas}: variáveis discretas que assumem seus valores em conjuntos finitos. Exemplo: as notas que um estudante pode obter em uma disciplina variam no conjunto {A, B, C, E}.
\vone
\textbf{Discretização}: consiste em atribuir um valor discreto diferente para variados intervalos de valores contínuos. Exemplo: as notas de uma disciplina de $0.0$
a $4.9$ poderiam ser consideradas como conceito E.

\end{ftst}

%=====

\begin{ftst}{O método de pesquisa}{Estruturação do trabalho científico}
\justifying
\textbf{Variáveis \textit{medidas} versus variáveis \textit{manipuladas}.}
\vone
Uma \textbf{variável medida} é aquela cujo fenômeno será observado pelo pesquisador. Exemplo: quantas vezes um usuário de uma ferramenta vai olhar no manual para obter informações para desempenhar a tarefa que lhe foi proposta. Essa variável tem como domínio o conjunto dos números naturais, e seus valores não são determinados pelo observador, mas simplesmente medidos.
\vone
Já a \textbf{variável manipulada} é aquela que o experimentador vai deliberadamente modificar para realizar seu experimento. Exemplo: o número de passos da tarefa repassada aos usuários.

\vone
O pesquisador manipularia a variável para tentar descobrir se tarefas mais longas implicam ou não o usuário consultar o manual do sistema mais vezes.


\end{ftst}

%=====

\begin{ftst}{O método de pesquisa}{Estruturação do trabalho científico}
\justifying
\textbf{Variáveis \textit{independentes} versus variáveis \textit{dependentes}.}
\vone
Exemplo: o número de passos em uma tarefa implica o aumento do número de vezes que o usuário consulta o manual?
\vone
Essa seria uma hipótese de pesquisa em que a variável dependente é o número de consultas ao manual, e a variável independente é o número de passos da tarefa.
\vone
Em geral, o pesquisador manipula a variável independente e mede a dependente. 

\end{ftst}

%=====

\begin{ftst}{A hipótese de pesquisa}{Estruturação do trabalho científico}
\justifying
Um aspecto que diferencia o trabalho científico do trabalho técnico é a existência de uma hipótese de pesquisa. 
\vone
A hipótese é uma afirmação da qual não se sabe a princípio se é verdadeira ou falsa. O trabalho de pesquisa consiste justamente em tentar provar a veracidade ou a falsidade da hipótese.
\vone
Um objetivo sem uma boa hipótese pode ser muito arriscado. O objetivo consiste em tentar produzir algum conhecimento que ainda não existe. Mas, se não houver uma boa hipótese para justificar esse objetivo, corre-se o risco de realizar a pesquisa sem obter resultados.  

\end{ftst}

%=====



\end{document}