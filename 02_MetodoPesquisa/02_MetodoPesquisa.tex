\documentclass[t]{beamer}
\usepackage{hyperref}
% Load general definitions
\input{preamble.tex}

% Specific definitions
\title[]{Metodologia Científica}
\subtitle[]{Métodos de Pesquisa}
\author[]{Patrícia Lucas\\{\footnotesize }}
\institute{Bacharelado em Sistemas de Informação \\ IFNMG  - Campus Salinas}
\date{\scriptsize Salinas\\Fevereiro 2021}

\begin{document}

% cover page
\setbeamertemplate{footline}{}
\begin{frame}

\begin{center}
\includegraphics[width=.15\textwidth]{}
\end{center}
  \titlepage
  \begin{tikzpicture}[remember picture,overlay]
  \node[anchor=south east,xshift=-5pt,yshift=5pt] at (current page.south east) {\tiny Versão 1.2021};
  \node[anchor=south west,yshift=0pt] at (current page.south west) {\includegraphics[width=.25\textwidth]{Logos/salinas_horizontal_jpg.jpg}};
  \end{tikzpicture}  
\end{frame}

% Main slides

\begin{ftst}{Referência}{Metodologia Científica}
\vone
\justifying
\begin{figure}
    \centering
    \includegraphics[scale=0.3]{Figuras/ref.jpg}
\end{figure}

WAZLAWICK, R. S. Metodologia de Pesquisa em Ciência da Computação. Rio de Janeiro: Campus, 2009.

\end{ftst}

%=====

\begin{ftst}{O que é pesquisa?}{Métodos de Pesquisa}
\vone
\justifying
O termo \textbf{“pesquisa”} pode referir-se a diversas atividades humanas, que vão desde a realização de pesquisas eleitorais até a pesquisa científica que busca aumentar o conhecimento humano sobre como o mundo funciona.
\vone
A pesquisa, no contexto científico, também pode ser classificada de acordo com diferentes critérios: \textcolor{blue}{natureza, objetivos ou procedimentos técnicos.} 
\vone
Nem sempre um trabalho de pesquisa limita-se a um único tipo. Além disso, alguns tipos de pesquisa podem ser a base para outros.

\end{ftst}

%=====

\begin{ftst}{Quanto a natureza da pesquisa}{Métodos de Pesquisa}
\vone
\justifying
O \textcolor{blue}{trabalho original} busca apresentar conhecimento novo a partir de observações e teorias construídas para explicá-las. Assume-se a nova informação como relevante quando ela tem implicação na forma como se entendem os processos e sistemas ou quando tem implicação prática na sua realização.
\vone
Já os \textcolor{blue}{resumos de assunto} buscam apenas sistematizar uma área de conhecimento, usualmente indicando sua evolução histórica e estado da arte.

\end{ftst}

%=====

\begin{ftst}{Quanto a natureza da pesquisa}{Métodos de Pesquisa}
\vone
\justifying
\textbf{Qualitativas:} surgiram nas ciências sociais e se ocupam de variáveis que não podem ser medidas, apenas observadas.
\vone
\begin{itemize}
    \item Em computação, são pesquisas que se caracterizam por ser um estudo aprofundado de um sistema em um ambiente onde ele está sendo usado, ou em alguns casos, onde se espera que o sistema seja usado.
\end{itemize}



\end{ftst}

%=====

\begin{ftst}{Quanto a natureza da pesquisa}{Métodos de Pesquisa}
\vone
\justifying
\textbf{Quantitativas:} surgiram nas ciências naturais, onde as variáveis observadas são poucas, objetivas e medidas em escalas numéricas.
\vone
\begin{itemize}
    \item As variáveis a serem observadas são consideradas objetivas, isto é, diferentes observadores obterão os mesmos resultados em observações distintas.
    \item Não há desacordo do que é melhor e o que é pior para os valores dessas variáveis objetivas.
    \item Medições numéricas são consideradas mais ricas que descrições verbais, pois elas se adéquam à manipulação estatística.
    \item Em computação, são pesquisa que se caracterizam por verificar o quão "melhor" é usar um programa/sistema novo frente às alternativas.
\end{itemize}

\end{ftst}

%=====

\begin{ftst}{Quanto aos objetivos da pesquisa}{Métodos de Pesquisa}
\vone
\justifying
A \textcolor{blue}{pesquisa exploratória} é aquela em que o autor não tem necessariamente uma hipótese ou objetivo definido em mente. 
\vone
Ela pode ser considerada, muitas vezes, como o primeiro estágio de um processo de pesquisa mais longo. 
\vone
Na pesquisa exploratória, o autor vai examinar um conjunto de fenômenos, buscando anomalias que não sejam ainda conhecidas e que possam ser, então, a base para uma pesquisa mais elaborada.
\vone


\end{ftst}

%=====

\begin{ftst}{Quanto aos objetivos da pesquisa}{Métodos de Pesquisa}
\vone
\justifying
A \textcolor{blue}{pesquisa descritiva} é mais sistemática do que a exploratória. Com ela busca-se obter dados mais consistentes sobre determinada realidade, mas não há ainda interferência do pesquisador ou a tentativa de obter teorias que expliquem os fenômenos. 
\vone
Tenta-se apenas descrever os fatos como são.
\vone
A pesquisa descritiva é caracterizada pelo levantamento de dados e pela aplicação de entrevistas e questionários. 
\vone
Assim como a pesquisa exploratória, ela pode ser considerada um passo prévio para encontrar fenômenos não explicados pelas teorias vigentes.

\end{ftst}

%=====

\begin{ftst}{Quanto aos objetivos da pesquisa}{Métodos de Pesquisa}
\vone
\justifying
A \textcolor{blue}{pesquisa explicativa} é a mais complexa e completa. 
\vone
É a pesquisa científica por excelência porque, além de analisar os dados observados, busca suas causas e explicações, ou seja, os fatores determinantes desses dados.


\end{ftst}

%=====

\begin{ftst}{Quanto aos procedimentos técnicos da pesquisa}{Métodos de Pesquisa}
\vone
\justifying
A \textcolor{blue}{pesquisa bibliográfica} implica o estudo de artigos, teses, livros e outras publicações usualmente disponibilizadas por editoras e indexadas.
\vone
A pesquisa bibliográfica é um passo fundamental e prévio para qualquer trabalho científico, mas ela em si não produz qualquer conhecimento novo. Apenas supre o pesquisador de informações públicas que ele ainda não possuía.


\end{ftst}

%=====

\begin{ftst}{Quanto aos procedimentos técnicos da pesquisa}{Métodos de Pesquisa}
\vone
\justifying
A \textcolor{blue}{pesquisa documental}, por outro lado, consiste na análise de documentos ou dados que não foram ainda sistematizados e publicados. Pode-se examinar relatórios de empresas, arquivos obtidos em órgãos públicos, bancos de dados, correspondências etc.
\vone
A pesquisa documental busca encontrar informações e padrões em documentos ainda não tratados sistematicamente.

\end{ftst}

%=====


\begin{ftst}{Quanto aos procedimentos técnicos da pesquisa}{Métodos de Pesquisa}
\vone
\justifying
A \textcolor{blue}{pesquisa experimental} caracteriza-se pela manipulação de um aspecto da realidade pelo pesquisador. 
\vone
O pesquisador introduz, por exemplo, uma nova técnica em uma empresa de software e observa se a produtividade aumenta. 
\vone
A pesquisa experimental implica ter uma ou mais variáveis experimentais que podem ser controladas pelo pesquisador (o fato de usar ou não determinada técnica, por exemplo), e uma ou mais variáveis observadas, cuja medição poderá levar, possivelmente, à conclusão de que existe algum tipo de dependência com a variável experimental.
\vone
A pesquisa experimental deve utilizar rigorosas técnicas de amostragem e testes de hipóteses para que seus resultados sejam estatisticamente aceitáveis e generalizáveis.

\end{ftst}

%=====

\begin{ftst}{Ciência versus tecnologia}{Métodos de Pesquisa}
\vone
\footnotesize
\justifying
Em computação, os termos ciência e tecnologia quase sempre andam tão juntos que muitas pessoas têm dificuldade em distingui-los. 
\vone
A ciência é a busca do conhecimento e das explicações, enquanto a tecnologia é a aplicação dos conhecimentos nas atividades práticas.
\vone
Observa-se que, algumas vezes, dissertações e teses em computação, bem como artigos científicos, ainda são fortemente caracterizados como apresentações meramente tecnológicas: sistemas, protótipos, frameworks, arquiteturas, modelos, processos, todas essas construções são técnicas, e não necessariamente
ciência.
\vone
Para que um trabalho seja efetivamente de cunho científico é necessário que a informação contida nele explique um pouco mais sobre o porquê de as coisas funcionarem como funcionam. 


\end{ftst}






\end{document}