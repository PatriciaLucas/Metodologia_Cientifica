\documentclass[t]{beamer}

% Load general definitions
\input{preamble.tex}

% Specific definitions
\title[]{Metodologia Científica}
\subtitle[]{Preparação do texto - Dicas}
\author[]{Patrícia Lucas\\{\footnotesize }}
\institute{Bacharelado em Sistemas de Informação \\ IFNMG  - Campus Salinas}
\date{\scriptsize Salinas\\Junho 2021}

\begin{document}

% cover page
\setbeamertemplate{footline}{}
\begin{frame}

\begin{center}
\includegraphics[width=.15\textwidth]{}
\end{center}
  \titlepage
  \begin{tikzpicture}[remember picture,overlay]
  \node[anchor=south east,xshift=-5pt,yshift=5pt] at (current page.south east) {\tiny Versão 1.2021};
  \node[anchor=south west,yshift=0pt] at (current page.south west) {\includegraphics[width=.25\textwidth]{Logos/salinas_horizontal_jpg.jpg}};
  \end{tikzpicture}  
\end{frame}

% Main slides

\begin{ftst}{Referência}{Preparação do texto - Dicas}
\vone
Materiais da Professora Mirella M. Moro da UFMG: \href{https://homepages.dcc.ufmg.br/~mirella/doku.php.}{\textcolor{blue}{link}}.

\vone
Guia para redação científica criado por Nathan Sheffield para a Escola de Pós-Graduação da Duke University : \href{https://sites.duke.edu/scientificwriting/}{\textcolor{blue}{link}}.

\end{ftst}

%=====

\begin{ftst}{Dicas}{Preparação do texto}
\justifying
\textbf{Nunca use:}
\vone

\begin{itemize}
    \item Frases longas (repletas de vírgulas ou não!).
    \item Frases incompreensíveis.
    \item Erros ortográficos.
    \item Tradução literal e "imbromation".
    \item Imagens/tabelas ilegíveis.
    \item Erros gramaticais (concordância, conjugação, crase).
    \item Não faça cópia literal. Quando referenciar outros trabalhos, resuma suas ideias principais.
    \item Frases que não dizem absolutamente nada de útil.
    \item Gírias, bem como ironias, brincadeiras, e referências pessoais ao leitor.
\end{itemize}

\end{ftst}

%=====

\begin{ftst}{Dicas}{Preparação do texto}
\justifying
\textbf{Evite:}
\vone

\begin{itemize}
    \item Uma seção formada apenas por uma lista de itens não é aceitável em um trabalho científico (artigo, monografia, etc.).
    \item Começar uma seção direto com uma subseção. É interessante que cada seção deve começar com pelo menos um parágrafo (nem que seja para introduzir o conteúdo das subseções).
    \item De mesmo modo, evite subseções que contêm apenas um parágrafo, sem introdução alguma. 
    \item Utilização de primeira pessoa, trocar pelo impessoal.
    \item Evitar “abaixo” e “acima”, pois não se pode garantir que essas informações estarão apresentadas na mesma página. Utilize “a seguir” e “anterior”.
\end{itemize}

\end{ftst}

%=====

\begin{ftst}{Dicas}{Preparação do texto}
\justifying
\textbf{Evite:}
\vone
\begin{itemize}
    \item Termos em inglês quando existe um termo técnico em português (performance $\rightarrow$ desempenho). De mesmo modo, evite traduzir pessoalmente conceitos que só existem em inglês. Se for inevitável, coloque o termo original em inglês entre parênteses e itálico.
\end{itemize}

\end{ftst}

%=====

\begin{ftst}{Dicas}{Preparação do texto}
\justifying
\textbf{Atenção!}
\vone

\begin{itemize}
    \item Confira cuidadosamente a seção de "Instruções a Autores"/"Instruções para Submissão"/"Regulamento e modelo de TCC".
    \item SQL, Sql ou sql? Como é uma sigla, utilize sempre a primeira opção (tudo em maiúsculo). Mais importante, utilize sempre da mesma forma em todo o texto.
    \item Seja consistente no use de tempo verbal.
    \item Palavras estrangeiras devem estar em itálico.
    \item Siglas esclarecidas – Colocar seu significado entre parênteses na primeira vez que aparecerem no texto.
\end{itemize}

\end{ftst}

%=====



\begin{ftst}{Dicas}{Preparação do texto}
\justifying
\textbf{Gráfico ou tabela?}
\vone
\begin{itemize}
    \item Se os dados mostram uma tendência, criando uma ilustração interessante, faça uma figura.
    \item Se os números apenas estão lá, sem qualquer tendência interessante em evidência, uma tabela deve ser suficiente.
    \item Tabelas também são preferíveis para apresentar números exatos.
\end{itemize}

\end{ftst}

%=====



\begin{ftst}{Dicas}{Preparação do texto}
\justifying
\textbf{Figuras, Gráficos e tabelas:}
\vone
\begin{itemize}
    \item Figuras possuem explicação detalhada no texto, já as tabelas/gráficos podem ser auto-suficientes.
    \item Se o trabalho apresenta um processo complicado, cheio de fases, entradas e saídas, tente resumir tudo em uma figura.
    \item As legendas das tabelas, gráficos e figuras devem ser auto-suficientes. 
    \item Se uma tabela, gráfico ou figura é usada, ela deve, obrigatoriamente, ser referenciada no texto.
\end{itemize}

\end{ftst}

%=====



\begin{ftst}{Dicas}{Preparação do texto}
\justifying
\textbf{Para finalizar:}
\vone
É uma boa ideia começar a escrever o artigo (ou monografia) enquanto o trabalho está em desenvolvimento (enquanto ideias, problemas, soluções e detalhes estão mais frescos na memória). 
\vone
Ler a Parte III do livro: Preparing the Tables and Figures.

\end{ftst}



\end{document}