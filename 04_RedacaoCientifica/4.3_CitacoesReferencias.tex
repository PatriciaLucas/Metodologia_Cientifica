\documentclass[t]{beamer}

% Load general definitions
\input{preamble.tex}

% Specific definitions
\title[]{Metodologia Científica}
\subtitle[]{Referências/Citações}
\author[]{Patrícia Lucas\\{\footnotesize }}
\institute{Bacharelado em Sistemas de Informação \\ IFNMG  - Campus Salinas}
\date{\scriptsize Salinas\\Junho 2021}

\begin{document}

% cover page
\setbeamertemplate{footline}{}
\begin{frame}

\begin{center}
\includegraphics[width=.15\textwidth]{}
\end{center}
  \titlepage
  \begin{tikzpicture}[remember picture,overlay]
  \node[anchor=south east,xshift=-5pt,yshift=5pt] at (current page.south east) {\tiny Versão 1.2021};
  \node[anchor=south west,yshift=0pt] at (current page.south west) {\includegraphics[width=.25\textwidth]{Logos/salinas_horizontal_jpg.jpg}};
  \end{tikzpicture}  
\end{frame}

% Main slides

%=====

\begin{ftst}{O que são?}{Referências/Citações}
\justifying
\textbf{O que é uma referência bibliográfica?} referência é o conjunto padronizado de elementos descritivos, retirados de um documento que permite sua identificação individual.
\vone
\textbf{O que é uma citação?} É informar para o leitor qual a fonte de uma determinada informação que você está se referindo no texto. Isso pode ser quando você está discutindo ou resumindo uma teoria ou informação com suas próprias palavras, ou pode estar citando diretamente dessa fonte.

\end{ftst}

%=====

\begin{ftst}{Exemplo}{Referências/Citações}
\justifying
\vone
\textbf{Citação:} "\textit{O método STN foi introduzido em \textbf{Song and Chissom (1993) }para lidar com o conhecimento vago e impreciso em dados de séries temporais.}"
\vone
\textbf{Referência:} \textit{Song, Q. and Chissom, B.S. (1993). Fuzzy time series andits models.Fuzzy Sets and Systems, 54(3), 269–277. doi:10.1016/
0165-0114(93)90372-O.}


\end{ftst}

%=====

\begin{ftst}{2 regras importantes}{Referências/Citações}
\justifying
\vone
\begin{itemize}
    \item[1.] usar referências \textbf{publicadas} e que sejam \textbf{significativas}. 
    \item[2.] garantir que todas as referências citadas no texto sejam realmente listadas na literatura citada e que todas as referências listadas na literatura citada sejam realmente citadas em algum lugar do texto.
\end{itemize}

\end{ftst}

%=====

\begin{ftst}{Tipos de citação}{Referências/Citações}
\justifying

\begin{itemize}
    \item Citação Direta: pode ser definida como a transcrição de um trecho completo da obra que está sendo consultada, ou seja, é a transcrição mais literal possível que o estudante utiliza e essa exige uma margem para ser usada no trabalho acadêmico em especial.
    \vone
    \item Citação Indireta: pode ser chamada também de paráfrase, essa é usada quando o estudante faz uma conexão do texto com outro autor, ou seja, faz o uso, porém com suas palavras.
    \vone
    \item Citação de Citação: quer dizer que seria uma citação usada de uma citação usada por outro autor(a).
\end{itemize}
\vone
\centering
\textcolor{red}{As citações Diretas e Citações de citações existem, mas não são usuais na área da Computação!}

\end{ftst}

%=====

\begin{ftst}{Estilos}{Referências/Citações}
\justifying
\begin{itemize}
    \item \textbf{Formato Harvard:} \href{https://www.mendeley.com/guides/harvard-citation-guide}{\textcolor{blue}{guia completo aqui}.}
    \vone
    \item \textbf{Formato Vancouver:} \href{https://guides.lib.monash.edu/citing-referencing/vancouver}{\textcolor{blue}{guia completo aqui}.}
    \vone
    \item \textbf{Formato IEEE:} \href{https://journals.ieeeauthorcenter.ieee.org/your-role-in-article-production/ieee-editorial-style-manual/}{\textcolor{blue}{guia completo aqui}.}
    \vone
    \item \textbf{Formato ABNT:} \href{https://www.ufpe.br/documents/40070/1837975/ABNT+NBR+6023+2018+\%281\%29.pdf/3021f721-5be8-4e6d-951b-fa354dc490ed}{\textcolor{blue}{guia completo aqui}.}
\end{itemize}



\end{ftst}


\end{document}