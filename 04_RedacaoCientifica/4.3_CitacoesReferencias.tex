\documentclass[t]{beamer}

% Load general definitions
\input{preamble.tex}

% Specific definitions
\title[]{Metodologia Científica}
\subtitle[]{Referências}
\author[]{Patrícia Lucas\\{\footnotesize }}
\institute{Bacharelado em Sistemas de Informação \\ IFNMG  - Campus Salinas}
\date{\scriptsize Salinas\\Junho 2021}

\begin{document}

% cover page
\setbeamertemplate{footline}{}
\begin{frame}

\begin{center}
\includegraphics[width=.15\textwidth]{}
\end{center}
  \titlepage
  \begin{tikzpicture}[remember picture,overlay]
  \node[anchor=south east,xshift=-5pt,yshift=5pt] at (current page.south east) {\tiny Versão 1.2021};
  \node[anchor=south west,yshift=0pt] at (current page.south west) {\includegraphics[width=.25\textwidth]{Logos/salinas_horizontal_jpg.jpg}};
  \end{tikzpicture}  
\end{frame}

% Main slides

\begin{ftst}{Referência}{Referências}

\justifying
\begin{figure}
    \centering
    \includegraphics[scale=0.35]{Figuras/ref2.jpg}
\end{figure}

Gastel, Barbara; Day, Robert A. How to Write and Publish a Scientific Paper. Califórnia: Greenwood, 2016.

\end{ftst}

%=====

\begin{ftst}{Dicas}{Referências}
\justifying
\textbf{2 regras importantes:}
\vone
\begin{itemize}
    \item[1.] listar apenas referências \textbf{publicadas} e que sejam \textbf{significativas}. 
    \item[2.] garantir que todas as referências citadas no texto sejam realmente listadas na literatura citada e que todas as referências listadas na literatura citada sejam realmente citadas em algum lugar do texto.
\end{itemize}

\end{ftst}



\end{document}