\documentclass[t]{beamer}

% Load general definitions
\input{preamble.tex}

% Specific definitions
\title[]{Metodologia Científica}
\subtitle[]{Redação Científica - Introdução}
\author[]{Patrícia Lucas\\{\footnotesize }}
\institute{Bacharelado em Sistemas de Informação \\ IFNMG  - Campus Salinas}
\date{\scriptsize Salinas\\Junho 2021}

\begin{document}

% cover page
\setbeamertemplate{footline}{}
\begin{frame}

\begin{center}
\includegraphics[width=.15\textwidth]{}
\end{center}
  \titlepage
  \begin{tikzpicture}[remember picture,overlay]
  \node[anchor=south east,xshift=-5pt,yshift=5pt] at (current page.south east) {\tiny Versão 1.2021};
  \node[anchor=south west,yshift=0pt] at (current page.south west) {\includegraphics[width=.25\textwidth]{Logos/salinas_horizontal_jpg.jpg}};
  \end{tikzpicture}  
\end{frame}

% Main slides

\begin{ftst}{Referência}{Redação Científica}

\justifying
\begin{figure}
    \centering
    \includegraphics[scale=0.35]{Figuras/ref2.jpg}
\end{figure}

Gastel, Barbara; Day, Robert A. How to Write and Publish a Scientific Paper. Califórnia: Greenwood, 2016.

\end{ftst}

%=====


\begin{ftst}{O escopo}{Redação Científica}
\justifying
O termo redação científica comumente inclui:
\vone
\begin{itemize}
    \item o relato de pesquisas originais em periódicos, por meio de artigos científicos em formato padrão.
    \item a comunicação sobre ciência por meio de outros tipos de artigos de periódicos, como artigos de revisão que resumem e integram pesquisas publicadas anteriormente.
    \item em um sentido ainda mais amplo, outros tipos de comunicação profissional por cientistas - por exemplo, propostas de bolsas, apresentações orais e apresentações de pôsteres.
\end{itemize}


\end{ftst}

%=====

\begin{ftst}{A necessidade de clareza}{Redação Científica}
\justifying
\textbf{A principal característica da redação científica é a clareza.}
\vone
A experimentação científica bem-sucedida é o resultado de uma mente clara atacando um problema claramente declarado e produzindo conclusões claramente definidas.
\vone
Idealmente, a clareza deve ser uma característica de qualquer tipo de comunicação; entretanto, quando algo está sendo dito pela primeira vez, clareza é essencial.
\vone
A maioria dos artigos científicos são aceitos para publicação precisamente porque contribuem com novos conhecimentos. Portanto, \textcolor{blue}{a clareza absoluta é uma exigência na redação científica}.

\end{ftst}

%=====


\begin{ftst}{Recebendo os sinais}{Redação Científica}
\justifying
Se uma árvore cai na floresta e não há ninguém para ouvi-la cair, ela faz algum barulho?
\vone
A resposta correta é não. O som é mais do que ondas de pressão e, de fato, não pode haver som sem um ouvinte.
\vone
Assim como um sinal de qualquer tipo é inútil a menos que seja percebido, um artigo científico publicado (sinal) é inútil a menos que seja recebido e compreendido por seu público-alvo.
\vone
\textbf{Conclusão:} um experimento científico não está completo até que os resultados tenham sido publicados e compreendidos.
\vone
Muitos artigos científicos caem silenciosamente na floresta!

\end{ftst}

%=====

\begin{ftst}{Entendendo os sinais}{Redação Científica}
\justifying
A escrita científica é a transmissão de um sinal claro a um destinatário. As palavras do sinal devem ser o mais claras, simples e bem ordenadas possível.
\vone
Na redação científica, há pouca necessidade de ornamentação. Enfeites literários floridos - metáforas, símiles, expressões idiomáticas - podem causar confusão e raramente devem ser usados em trabalhos de pesquisa.
\vone
A ciência é simplesmente importante demais para ser comunicada em qualquer outra coisa senão em palavras de significado certo. 
\vone
Muitos tipos de escrita são projetados para entretenimento. \textbf{A escrita científica tem um propósito diferente: comunicar novas descobertas científicas.}



\end{ftst}

%=====

\begin{ftst}{Entendendo o contexto}{Redação Científica}
\justifying
O que é claro para um destinatário depende tanto do que é transmitido quanto de como o destinatário o interpreta.
\vone
Portanto, comunicar-se claramente requer consciência do que o destinatário traz. Qual é o histórico do destinatário? O que o destinatário está procurando? Como o destinatário espera que a redação seja organizada? Clareza na redação científica requer atenção a tais questões.
\vone
Conheça o seu público! Conheça também as convenções e, portanto, as expectativas para estruturar o tipo de redação que você está fazendo.



\end{ftst}

%=====

\begin{ftst}{Organização e linguagem}{Redação Científica}
\justifying
Uma organização eficaz é a chave para se comunicar de forma clara e eficiente em ciência. 
\vone
Essa organização inclui seguir o formato padrão de um artigo científico. Também inclui organizar ideias de forma lógica dentro desse formato. 
\vone
Além da organização, o segundo ingrediente principal de um artigo científico deve ser a linguagem apropriada.
\vone
Dado que o resultado final da pesquisa científica é a publicação, é surpreendente que muitos cientistas negligenciem as responsabilidades envolvidas. 

\end{ftst}

%=====

\begin{ftst}{Organização e linguagem}{Redação Científica}
\justifying
Um cientista gastará meses ou anos de trabalho árduo e, então, despreocupadamente, permitirá que muito de seu valor seja perdido devido à falta de interesse no processo de comunicação.
\vone
“O melhor inglês/português é aquele que dá sentido no menor número de palavras curtas”.
\vone
Dispositivos literários, metáforas e coisas semelhantes, desviam a atenção da substância para o estilo. Eles devem ser usados raramente em redação científica.

\end{ftst}


\end{document}