\documentclass[t]{beamer}

% Load general definitions
\input{preamble.tex}

% Specific definitions
\title[]{Metodologia Científica}
\subtitle[]{Introdução}
\author[]{Patrícia Lucas\\{\footnotesize }}
\institute{Bacharelado em Sistemas de Informação \\ IFNMG  - Campus Salinas}
\date{\scriptsize Salinas\\Fevereiro 2021}

\begin{document}

% cover page
\setbeamertemplate{footline}{}
\begin{frame}

\begin{center}
\includegraphics[width=.15\textwidth]{}
\end{center}
  \titlepage
  \begin{tikzpicture}[remember picture,overlay]
  \node[anchor=south east,xshift=-5pt,yshift=5pt] at (current page.south east) {\tiny Versão 1.2021};
  \node[anchor=south west,yshift=0pt] at (current page.south west) {\includegraphics[width=.25\textwidth]{Logos/salinas_horizontal_jpg.jpg}};
  \end{tikzpicture}  
\end{frame}

% Main slides
\begin{ftst}{Ementa}{Metodologia Científica}
\vone
\vone
\begin{itemize}
    \item[1.] Bases filosóficas do método científico. 
    \item[2.] Estruturação do trabalho científico: planos e projetos de trabalho. 
    \item[3.] Pesquisa e organização das fontes de referência bibliográfica e citação. 
    \item[4.] Elaboração, revisão, edição e apresentação do trabalho científico.
\end{itemize}

\end{ftst}

%=====

\begin{ftst}{Bibliografia básica}{Metodologia Científica}
\vone
\vone
\begin{itemize}
    \item[1.] WAZLAWICK, R. S. Metodologia de Pesquisa em Ciência da Computação. Rio de Janeiro: Campus, 2009.
    \item[2.] APPOLINÁRIO, F. Metodologia da Ciência: Filosofia e Prática da Pesquisa. 2. ed. Cengage Learning, 2012.
    \item[3.] LAKATOS, E. M.; MACONI, M. A. Fundamentos de Metodologia Científica. 7. ed. São Paulo: Atlas, 2010.
\end{itemize}

\end{ftst}

%=====

\begin{ftst}{Bases filosóficas do método científico}{Metodologia Científica}
\vone
\vone
\begin{itemize}
    \item O método científico.
    \item Métodos de pesquisa.
\end{itemize}

\end{ftst}

%=====

\begin{ftst}{Estruturação do trabalho científico}{Metodologia Científica}
\vone
\vone
\begin{itemize}
        \item Escolher o objetivo da pesquisa.
        \item A revisão bibliográfica.
        \item O objetivo.
        \item O método de pesquisa.
        \item A hipótese de pesquisa.
        \item Justificativa da hipótese.
        \item Resultados esperados.
        \item Limitações do trabalho.
        \item Discussão.
\end{itemize}

\end{ftst}

%=====

\begin{ftst}{Referências bibliográficas}{Metodologia Científica}
\vone
\vone
\begin{itemize}
    \item Pesquisa de referências bibliográficas.
    \item Ferramentas para organização de referências bibliográficas.
\end{itemize}
\end{ftst}

%=====

\begin{ftst}{Elaboração de trabalho científico}{Metodologia Científica}
\vone
\vone
\begin{itemize}
    \item Como escrever o resumo.
    \item Como escrever a introdução.
    \item Como escrever a seção de metodologia.
    \item Como escrever a seção de resultados.
    \item Como escrever a seção de discussões.
    \item Como escrever a conclusão.
    \item Ferramenta de elaboração de trabalhos científicos em Latex.
\end{itemize}


\end{ftst}

%=====

\begin{ftst}{Apresentação de trabalho científico}{Metodologia Científica}
\vone
\vone
\begin{itemize}
    \item Apresentação de um trabalho científico.
\end{itemize}


\end{ftst}

%=====



\end{document}

